% ■ アブストラクトの出力 ■
%	◆書式:
%		begin{jabstract}〜end{jabstract}	:日本語のアブストラクト
%		begin{eabstract}〜end{eabstract}	:英語のアブストラクト
%		※ 不要ならばコマンドごと消せば出力されない。



% 日本語のアブストラクト
\begin{jabstract}

医療情報は秘匿性の高い情報であると同時に患者の既往歴や持病、服薬といった情報は医療対応上で重要度の高い情報である。
従来の医療情報共有は紙面に頼ることが多く、開示すべき対象や情報を制限することが難しい。本研究では救急時、難病患者の医療情報を適切な範囲で本人の行動を必要とせずに選択的に開示を可能とし、患者のプライバシーを保護しながら適切な対象に対して、適切な範囲の情報を提供することを目的とする。

\end{jabstract}



% 英語のアブストラクト
\begin{eabstract}

Medical information is highly confidential information, and at the same time, information such as the patient's medical history, chronic condition, and medication is highly important for medical treatment.
Conventional medical information sharing often relies on paper, and it is difficult to limit the objects and information to be disclosed. In this study, medical information of patients with intractable diseases can be selectively disclosed within an appropriate range without the need for their own actions during an emergency, and the appropriate range is provided for an appropriate target while protecting the privacy of the patient. The purpose is to provide information.

\end{eabstract}
