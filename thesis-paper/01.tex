\chapter{序論}
\label{chap:introduction}





\begin{comment}
論文は序論のようなもので始める。タイトルは序論でも序言でもはじめにでもいいけど、『序論』で始めたら『結論』で終わり、『序言』で始めたら『結言』で終わるようにする。『はじめに』なら『おわりに』で終わる。『序論』で始まって『おわりに』でおわるとか、そういうちぐはぐなのはだめ。

ここでは序論として書く。序論では、研究の背景やら目的やらを書くのが普通。今はテンプレートの説明なので、大して書くことは無い。
\end{comment}



\section{背景}

ここでは難病患者の定義や、医療情報共有に関する従来の方式について説明する。

\begin{quotation}

近年では環境や行動に制限や危険を多く持つ難病等の持病を持つ患者もアクティブに旅行をすることが増え、それに伴い危険行動を適切に認識することや、医療体制を整える必要性があるとされている。
厚生労働省は難病について1.発病の機構が明らかでない。2.治療法が確立されていない。3.希少な病気。といった定義づけを行なっており、この定義から難病患者は難病を持たない患者に比べ危険行動が多く、一般での理解もされていないと言える。そのため患者は自身の医療情報を紙媒体等で持ち運ぶことや、体調不良時自身の持病や服薬について医療関係者に説明すること、患者自身が意識不明になった時を想定し事前に家族や友人、さらには旅行の同行者等に自身の持病の開示と危険行動の説明をすることが必要とされる。

\end{quotation}



\begin{comment}
筆者はこの卒業論文用のテンプレートを大学院ガイドに例示されている体裁\cite{mag_guide12}に沿うように改造した。これは論文の形式で言えばもっと後ろに書いてあるべきことなのかもしれない。
\end{comment}


\section{本文書の構成}

第\ref{chap:introduction}章では本研究の背景について述べた。第\ref{chap:problem}章では、具体的に解決すべき問題点について書いた。第\ref{chap:propose}章では解決のために必要な要件を挙げ、実際にどのようにして実現するかの手法について述べた。第\ref{chap:implementation}章で実装をし、評価を行なった。第\ref{chap:conclusion}章では上記を踏まえ、得た結論についてまとめた。